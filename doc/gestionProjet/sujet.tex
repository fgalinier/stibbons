%initial/final : évolution
Le sporjet Stibbons a pour but la création d'un programme capable d'analyser un langage dérivé de Logo, le Stibbons, et d'afficher l'animation des «tortues», les agents que l'utilisateur programme. Ce projet a pour but de pouvoir effectuer des simulations de comportement. Par exemple, un exemple connu est celui des thermites : l'utilisateur programme un comportement pour ses agents, ses tortues, fait afficher des bindilles sur le sol (les zones) et observe les agissements des tortues.
//IMAGE DE LA SIMULATION DES THERMITES SOUS NETLOGO

 Au début du projet, nous devions avoir une application graphique comparable à celle de NetLogo.
IMAGE DE NETLOGO.


Au cours du projet, notre tuteur, Michel Meynard, nous a sugéré de rajouter l'exportation des données que nous manipulions (le modèle, qui comprend l'ensemble des tortues, les zones et le monde). Nous avons donc rajouté cette possibilité à notre application.
Plus tard, nous avons aussi imaginé un programme complémentaire à notre application : un lancement à la console du programme, sans affichage graphique du déroulement de la simulation, mais avec des captures de cette simulation enregistrer sous forme d'image et de JSON, pour les données.
Cela permet, lorsqu'on fait une simulation longue de la laisser tourner sans ralentir l'ordinateur et en ayant une idée visuel de l'évolution des tortues durant la simulation. En outre, ce programme permet d'executer plus rapidement la simulation. POURQUOI ?

