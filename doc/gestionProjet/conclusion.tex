Au cours de ces quatre mois, nous avons beaucoup appris sur SCRUM et l'Agile, grâce au projet mais également grâce à Sandrine Maton, intervenante extérieure à l'UM.

Le planning a été respecté, avec des sprints réguliers de deux semaines tout en faisant évoluer nos objectifs.
Au cours de cette évolution, nous avons senti une réelle cohésion de groupe se former, notamment à partir du troisième sprint où le fait de travailler avec une méthode agile permis de mieux s'organiser et nous incita fortement à communiquer.

L'ajout de tâches au backlog n'a pas perturbé notre rythme de développement et nous avons fini notre projet dans les temps, avec quelques fonctionnalités supplémentaires initialement non prévues.

De plus nous avons des idées d'améliorations à proposer pour une potentielle version 2.0.
