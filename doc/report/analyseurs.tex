L'application Stibbons fournit un interprète (ou interpréteur) pour le langage du même nom. Cette interprétation du code se déroule en deux phases~: une de compilation lors du chargement du code, et une d'interprétation de l'arbre abstrait (généré lors de la première phase) lors de l'exécution.

La première phase peut elle-même être découpée en deux parties~:
\begin{itemize}
\item l'analyseur lexical qui «~lit le flot de caractères qui constituent le programme source et les regroupe en séquences de caractères significatives appelées \emph{lexèmes}.~» \cite{compilateurs}~;
\item l'analyseur syntaxique qui, à partir des lexèmes, génère un arbre abstrait qui pourra par la suite être analysé pour être interprété.
\end{itemize}

L'interprétation quant à elle se déroule lors de l'analyse sémantique de l'arbre abstrait, qui exécute les opérations contenues dans les nœuds.

\section{Analyseur lexical}
\label{analyse-lexicale}
L'analyse lexicale vise à produire un flot de jeton qui pourront être analyser par l'analyseur syntaxique. Ces jetons sont des paires composées d'un type de jeton et de la valeur du lexème (par exemple, l'analyse du lexème \verb|12| va générer le jeton \verb|<NUMBER,12>| dans notre cas). Certains lexèmes peuvent générer des jetons qui n'ont pas de valeur (par exemple, le lexème \verb|(| va entrainer la génération du jeton \verb|<(>|).

Nous avons fait le choix d'utiliser l'outil Flex pour notre projet (cf \ref{Flex}).

\subsection{Jeton}


\subsection{Règles}


L'analyseur lexical de Stibbons est écrit en Flex et permet de détecter certaines chaînes de caractères pour y générer des séquences correspondantes dans l'analyseur syntaxique.

Par exemple, lors de la détection d'un mot (prenons le mot \verb|recv|), la séquence correspondante est générer (ici, la séquence RECV serait produite).
Un comptage de ligne pour la gestion d'erreur y est également faite.

%%Et là en dessous c'est un copié collé de la partie de julia

\subsubsection{Théorie}
Pour faire la compilation, la première étape est l'analyse du fichier source.
Tout d'abord, on fait une analyse d'un point de vue lexicale, c'est à dire qu'on décompose les chaînes de caractère (le code) en lexème ou jeton.\\
L'une des façons de faire est de construire un automate à état fini associé au mots reconnus.\\
Suite à la reconnaissance d'un mot ou lexème, l'analyseur lexicale retourne un jeton correspondant  à la catégorie lexicale du lexème. Plus précisément, on retourne un couple (jeton, valeur sémantique).\\
Par exemple, si on définit  (if , 300) et qu'on reconnaît un if on retourne (IF,).
On choisit plutôt des entiers pour représenter les catégories lexicales.\\
Pour les variables, il faut retourner une valeur sémantique, donc soit le lexème lui-même pour un littéral entier, l'indice d'entrée correspondant dans la table des symboles pour une variable.\\
Par exemple : (LITTERALCHAINE,'Bonjour !').\\
Pour éviter les erreurs avec les mots préfixes d'autres, on applique la règle du mot le plus long : on regarde le caractère suivant, s'il étend le lexème reconnu, on continue.\\
Il faut pouvoir revenir en arrière si on a été trop loin dans la lecture et qu'on se retrouve dans un état non terminal. Il faut donc connaître les états de notre automate.
Il faut aussi penser à filtrer les blancs et les commentaires, selon la grammaire.
Un générateur d'analyseur permet d'éviter cette étape.\\



\section{Analyseur syntaxique}
\label{analyse-syntaxique}
L'analyse syntaxique permet de vérifier que la structure d'un programme est bien en accord avec les règles de grammaire du langage. Par exemple, la grammaire de notre langage comporte une règle qui indique que le jeton \verb|<FD>| doit être suivi d'un jeton \verb|<NUMBER,>|. Le but de notre analyse ici est double~: vérifier que le programme est bien un programme de notre langage valide, mais également généré un arbre abstrait qui pourra facilement être analysé par notre analyseur sémantique.

Nous avons fait le choix d'utiliser GNU Bison (cf.~\ref{bison}) dans notre projet.

\subsection{Arbre abstrait}
Un arbre abstrait est une structure d'arbre dont chaque noeud feuille représente les opérandes des opérations contenues sur les autres noeuds. Cet arbre peut être considéré comme une forme de code intermédiaire, et peut être interprété très facilement en interprétant pour chaque noeud, les sous-arbres et en appliquant l'opération du noeud courant.

\begin{lstlisting}[language=Stibbons,label=arbre-code,caption=Exemple de code Stibbons]
  agent turtle (a) {
    teleport(50,50,0)
    fd a
  }

  b = 18
  repeat b {
    if(b == 18) {
      new turtle(b)
    }
    else {
      new turtle(3)
    }
    b = b - 1
  }
\end{lstlisting}

\begin{figure}[h]
\centering
\includegraphics[scale=0.8]{doc/report/img/arbre-abstrait}
\caption{\label{arbre-abstrait} Arbre abstrait généré lors de l'analyse du code \ref{arbre-code}}
\end{figure}

L'avantage de générer un tel arbre est qu'il est bien plus rapide d'effectuer un parcours d'arbre à chaque interprétation plutôt que de refaire une analyse syntaxique à chaque fois.

\subsection{Fonctionnement}

\subsubsection{Analyse syntaxique LALR}


\section{Analyseur sémantique}
\label{Analyseur sémantique}
L'analyse sémantique du Stibbons est effectué dans les classes Interpreter. Plus exactement dans 3 classes. Voir l'UML~\ref{interpreterUML} correspondant page~\pageref{interpreterUML}.

\begin{figure}[h]
\caption{\label{interpreterUML} UML de l'analyseur sémantique}
\includegraphics[scale=0.5]{doc/report/uml/interpreterUML.png}
\end{figure}

On a donc les classes \verb|TurtleInterpreter| et \verb|WorldInterpreter| qui héritent de la classe Interpreter. Cette dernière analyse et provoque, dans le modèle, les actions liées au code stibbons écrit. Elle permet de gérer tout type d'interaction, du moment que ces actions ne sont pas liées à un monde ou une tortue. La classe \verb|TurtleInterpreter|, gère justement ce dernier type d'actions : celles liées à une tortue. Contrairement à \verb|WorldInterpreter| qui gère les actions du monde.

L'utilité de la classe \verb|InterpreterManager| est expliqué de manière détaillée dans la section \ref{remaniementInterpreter} (page~\pageref{remaniementInterpreter}).

Ainsi, pour avoir un exemple concret, si on demande à une tortue d'avancer en stibbons (ex : \verb|fd 10|), alors c'est le \verb|TurtleInterpreter| qui gèrera cette action.
A contrario, si on effectue une définition de fonction (ex : \verb|a = 10|) alors l'\verb|Interpreter| affectera la valeur \verb|10| à la propriété \verb|a| de l'agent courant.
Pour ce qui est du WorldInterpreter, il n'y a pas encore d'exemple possible, tout simplement car pour l'instant il n'y a pas d'actions spéciales pour le monde.

\subsection{Fonctionnalités}

\subsubsection{Sprint 1 \& 2}
Les deux premiers sprints ont été assez conséquent au niveau du nombre de fonctionnalités ajoutées.
En effet lors du sprint 1 on pouvait déjà effectuer les opérations de bases sur une tortue, telles que avancer, tourner, écrire, etc. De plus les opérations de calculs ainsi que la gestions de nombre ont été fait. En effet, il était nécessaire de gérer les nombres de manière à pouvoir indiquer à la tortue de quelle distance elle devait avancer.

Lors du deuxième sprint sont apparu les conditionnelles, ainsi que les boucles, les comparaisons binaires, les booléens et autres type (color, string, etc.), la création de nouveaux agents dans le code et les fonctions sans paramètres. Ce sprint fut alors une version déjà bien avancé de notre programme final.


\subsubsection{Sprint 3 à 5}
Les sprints suivants ont été plus léger. Non pas parce qu'il y avait moins de travail, car la gestion de l'interpréteur fut remanier à ce moment là, mais car il y avait moins de fonctionnalités à ajouter. En effet à partir du sprint 3, les fonctionnalités suivantes ont été rajoutés~:
\begin{itemize}
\item accès à la parenté et aux zones (en lecture seulement)~;
\item ajout des tables et de boucles dédiés (\verb|foreach|)~;
\item gestion de la vitesse et de la pause~;
\item ajout des communications avec le \verb|send| et le \verb|recv|.
\end{itemize}


\subsection{Remaniement de l'analyseur sémantique}
\label{remaniementInterpreter}

L'analyseur sémantique a d'abord été une seule classe~: la classe \verb|Interpreter|.
Cette dernière implémentait tout types d'actions a effectué pour n'importe quel type d'agent.
Cependant, lors de l'arrivé de la fonctionnalité d'ajout d'un nouvel agent (\verb|new agent|), nous nous sommes aperçu qu'il serait mieux d'avoir un interpréteur par type d'agent, ou plus précisement un interpréteur pour le monde, un pour les tortues et un pour les actions communes aux deux types.
Nous avons alors crée les deux classes~: \verb|WorldInterpreter| et \verb|TurtleInterpreter|.

De manière parallèle, la création du monde s'effectuait dans l'\verb|Interpreter|, puis dans le \verb|WorldInterpreter|, il fut alors nécessaire de crée une classe qui gérerait à la fois la création du monde et tout ce qui concernait l'application (la pause, le temps, les interpreteurs eux-mêmes). Nous avons alors décidé de crée la classe \verb|InterpreterManager|.
Cette classe connait ainsi tout les interpréteurs, permet de les prévenir d'une éventuelle pause du programme, de stocker les threads correspondants à chaque interpréteur, de crée un monde avec les pré-directives choisies~; c'est une sorte de gestionnaire d'interpréteurs.

