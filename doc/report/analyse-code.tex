\section{Analyse de code}

\subsection{Flex}
\subsubsection{Théorie}
Pour faire la compilation, la première étape est l'analyse du fichier source.
Tout d'abord, on fait une analyse d'un point de vue lexicale, c'est à dire qu'on décompose les chaînes de caractère (le code) en lexème ou jeton.\\
L'une des façons de faire est de construire un automate à état fini associé au mots reconnus.\\
Suite à la reconnaissance d'un mot ou lexème, l'analyseur lexicale retourne un jeton correspondant  à la catégorie lexicale du lexème. Plus précisément, on retourne un couple (jeton, valeur sémantique).\\
Par exemple, si on définit  (if , 300) et qu'on reconnaît un if on retourne (IF,).
On choisit plutôt des entiers pour représenter les catégories lexicales.\\
Pour les variables, il faut retourner une valeur sémantique, donc soit le lexème lui-même pour un littéral entier, l'indice d'entrée correspondant dans la table des symboles pour une variable.\\
Par exemple : (LITTERALCHAINE,'Bonjour !').\\
Pour éviter les erreurs avec les mots préfixes d'autres, on applique la règle du mot le plus long : on regarde le caractère suivant, s'il étend le lexème reconnu, on continue.\\
Il faut pouvoir revenir en arrière si on a été trop loin dans la lecture et qu'on se retrouve dans un état non terminal. Il faut donc connaître les états de notre automate.
Il faut aussi penser à filtrer les blancs et les commentaires, selon la grammaire.
Un générateur d'analyseur permet d'éviter cette étape.\\

\subsection{Bison}

Yacc est un outil d'analyse syntaxique. L'analyse syntaxique permet de vérifier qu'un mot appartient bien au langage.\\ Il génère un analyseur syntaxique ascendant utilisant un automate à pile (dérivation à droite, on remplace le symbole non terminal le plus à droite).
 Son fonctionnement est le suivant : à chaque règle de grammaire, on associe des actions (instructions d'un langage). L'analyseur généré essaie de reconnaître un mot du langage défini par la grammaire. Il exécute les actions pour chaque règle reconnue.  Bison est une version de yacc.\\

\textbf{Exemple} :\\
D'après une grammaire ambiguë, on construit un vérificateur syntaxique. On écrit un source yacc : fichier.y, dans lequel on définit :

 \begin{verbatim}%{ \end{verbatim}  \textit{déclaration C}  \begin{verbatim}%} \end{verbatim}
 \begin{verbatim}%%\end{verbatim} \textit{définition de la grammaire reconnu} \begin{verbatim}%% \end{verbatim}
\textit{définition fonction C}\\

Les définitions de fonction C  doit avoir une fonction main, qui appelle yyparse(), une définition de yylex() appellée par yyparse() et une définition de yyerror(char*) pour signaler un erreur à l'utilisateur.
On compile avec bison : \textit{bison -y fichier.y} , puis \textit{gcc -o fichier.y.tab.c}, puis lancer l’exécutable.\\
Pour utiliser C++ avec bison, on écrit en C++ dans la partie action. On aura du C et C++ dans l'analyseur : fonction C, et action en C++.\\

Si on veut un analyseur syntaxique en C++, il faut utiliser un squelette de parseur C++ en utilisant soit l'option bison -skeleton=lalr1.cc ; soit en utilisant la directive \%skeleton « lalr1.cc ».\\ Ne pas oublier de déclarer yylex() aprés \%union !\\
\textit{Pour plus de détails, regarder le cours de Michel Meynard, dont ces informations sont principalement tirés.}
