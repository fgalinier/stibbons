\section{24 février 2015}

Lors de cette réunion nous avons tout d'abord fait un point sur le backlog et les taches que nous avions choisi pour le premier sprint~:
\begin{itemize}
\item interprète simple, qui ne comprend que des instructions basiques~;
\item une seule tortue est gérée~;
\item pas d'éditeur intégré dans un premier temps mais seulement lecture de code dans un fichier externe.
\end{itemize}
Nous avons également précisé que le code analysé serait lu en intégralité et interprété, et non pas interprété ligne à ligne, tout du moins dans un premier temps.
Ainsi, nous passerons par une phase de pré-compilation afin d'analyser le code source stibbons. Nous aurons donc un schéma du type~:
code source $\rightarrow$ tokenizer $\rightarrow$ analyse et interprétation

Nous définirons dans un premier temps un langage élémentaire, contenant les affectations ainsi que les instructions de bases pour les tortues (forward, turn-right, etc.) et augmenterons la complexité du langage dans le temps.
Il faudra ainsi définir très rapidement une grammaire, l'idéal étant pour la fin de la semaine. Cette tache est assigné à Florian et Clément.

Pendant ce temps-là, Adrien et Julia sont chargés de définir le modèle. On utilisera pour cela le langage UML, ainsi que le logiciel Umbrello qui permet la génération de code.

Nous avons également abordé la question de la gestion de la bibliographie. Nous utiliserons bibtex dans le rapport pour celle-ci.
