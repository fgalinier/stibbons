\section{Propriétés standard}

Les agents possèdent certaines propriétés par défaut. Ces propriétés varient selon leur type, et peuvent être de simples attributs prédéfinis, des fonctions prédéfinies, ou encore des propriétés spéciales ayant une sémantique particulière.

Les propriétés spéciales ont une sémantique particulière et peuvent par conséquent être en lecture seule ou avoir certaines requêtes quant au valeurs qui lui sont affectées.

\subsection{Attributs communs aux tortues et au monde}

\begin{description}
	\item[black] $\rightarrow$ \#000000
	\item[white] $\rightarrow$ \#ffffff
	\item[grey] $\rightarrow$ \#7f7f7f
	\item[red] $\rightarrow$ \#ff0000
	\item[green] $\rightarrow$ \#00ff00
	\item[blue] $\rightarrow$ \#0000ff
	\item[yellow] $\rightarrow$ \#ffff00
	\item[cyan] $\rightarrow$ \#00ffff
	\item[magenta] $\rightarrow$ \#ff00ff
\end{description}

\subsection{Fonctions communes aux tortues et au monde}

\begin{description}
	\item[print] (value) $\rightarrow$ null

	Imprime une valeur sur la sortie standard.

	\begin{description}
		\item[value] La valeur à imprimer
		\item[Retourne] null
	\end{description}

	\item[println] (value) $\rightarrow$ null

	Imprime une valeur dans une nouvelle ligne sur la sortie standard.

	\begin{description}
		\item[value] La valeur à imprimer
		\item[Retourne] null
	\end{description}

	\item[rand] () $\rightarrow$ number

	Retourne un entier positif au hasard.

	\begin{description}
		\item[Retourne] un entier positif au hazard
	\end{description}
\end{description}

\subsection{Attributs du monde}

\begin{description}
	\item[max\_x] $\rightarrow$ number

	L'abscisse maximale du monde.

	\emph{Attribut en lecture seule.}

	\item[max\_y] $\rightarrow$ number

	L'ordonnée maximale du monde.

	\emph{Attribut en lecture seule.}
\end{description}

\subsection{Attributs des tortues}

\begin{description}
	\item[color] $\rightarrow$ \#000000

	La couleur de la tortue.

	\emph{Requiert une valeur de type color.}

	\item[parent] $\rightarrow$ agent

	L'agent qui a créé la tortue.

	\emph{Attribut en lecture seule.}

	\item[pos\_x] $\rightarrow$ number

	L'abscisse de la position de la tortue.

	\emph{Requiert une valeur de type number.}

	\item[pos\_y] $\rightarrow$ number

	L'ordonnée de la position de la tortue.

	\emph{Requiert une valeur de type number.}

	\item[pos\_angle] $\rightarrow$ number

	L'angle de la tortue.

	\emph{Requiert une valeur de type number.}

	\item[world] $\rightarrow$ world

	Le monde.

	\emph{Attribut en lecture seule.}

	\item[zone] $\rightarrow$ zone

	La zone dans laquelle est la tortue.

	\emph{Attribut en lecture seule.}
\end{description}

\subsection{Fonctions de tortues}

\begin{description}
	\item[distance\_to] (turtle) $\rightarrow$ number

	Retourne la distance la plus courte vers une autre tortue.

	\begin{description}
		\item[turtle] La tortue vers laquelle obtenir la distance
		\item[Retourne] null
	\end{description}

	\item[face] (turtle) $\rightarrow$ null

	Tourne la tortue pour qu'elle fasse face à une autre par le chemin le plus court.

	\begin{description}
		\item[turtle] La tortue à laquelle faire face
		\item[Retourne] null
	\end{description}

	\item[in\_radius] (distance) $\rightarrow$ table

	Retourne l' ensemble de tortues dans le rayon donné autour de la tortue.

	\begin{description}
		\item[distance] Le rayon autour de la tortue à sonder
		\item[Retourne] Une table contenant les tortues dans le rayon
	\end{description}

	\item[inbox] () $\rightarrow$ number

	Retourne le nombre de message non lus.

	\begin{description}
		\item[Retourne] Le nombre de messages non lus
	\end{description}

	\item[teleport] (x, y, angle) $\rightarrow$ null

	Téléporte une tortue à une certaine coordonnée et avec un certain angle.

	\begin{description}
		\item[x] L'abscisse où téléporter la tortue
		\item[y] L'ordonnée où téléporter la tortue
		\item[angle] L'angle à donner à la tortue
		\item[Retourne] null
	\end{description}
\end{description}

\subsection{Attributs des zones}

\begin{description}
	\item[color] $\rightarrow$ \#ffffff

	La couleur de la zone.

	\emph{Requiert une valeur de type color.}
\end{description}

