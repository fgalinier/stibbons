\section{Analyseur lexical}

L'analyseur lexical de Stibbons est écrit en Flex et permet de détecté certaines chaînes de caractères pour y générer des séquences correspondantes dans l'analyseur syntaxique.

Par exemple, lors de la détéction d'un mot (prenons le mot \verb|recv|), la séquence correspondant est générer (ici, la séquence RECV serait produite).
Un comptage de ligne pour la gestion d'erreur y est également faite.

%%Et là en dessous c'est un copié collé de la partie de julia
\subsubsection{Pratique}
Flex est une version libre de l'analyseur lexical Lex. Il est généralement associé à l'analyseur syntaxique GNU Bison, la version GNU de Yacc. \\
Il lit les fichiers d'entrée donnés pour obtenir la description de l'analyseur à générer. La description est une liste de paires d'expressions rationnelles et de code C, appelées règles. \\
Flex a plusieurs régles pour l'écriture du fichier .l+ .\\
D'abord, un espace entre \%\{ \%\} qui contient une partie optionnelle de définition.\\
Par exemple :
\ \begin{verbatim}
%{
include <iostream.h>
class A\{ 
 public :
  void Hello() {cout<< ``Hello world !''<<endl ; }
};
%}
\end{verbatim}
Une partie obligatoire de régles lex commencant par \%\%.\\
Cette partie associe des instructions C++ à des expressions régulières.
\begin{verbatim}
%%
[a-z]([a-z])*           {return 5;}
\end{verbatim}
Enfin, une partie optionnelle pour des fonctions C++ définis par l'utilisateur, commençant par \%\%.\\
\begin{verbatim}
%%
int main(){
...
}
\end{verbatim}
Pour compiler : \textit{flex -+ exemple.l+}  puis \textit{g++ -g -o exemple lex.yy.cc -lfl}.\\
Après, c'est la fonction de l'analyseur syntaxique yyparse() qui appelle yylex() pour avoir les jetons correspondant au fichier lu.\\
La fonction main() appelle yyparse().\\
