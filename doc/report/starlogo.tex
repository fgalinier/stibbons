\section{StarLogo}
\label{StarLogo}
Tout comme NetLogo, StarLogo est un environnement de modélisation programmable permettant de simuler et d'observer des phénomènes naturels et sociaux au fil du temps.
Ils ont les mêmes objectifs d'étude, d'éducation et de «~programmation facile~» ainsi qu'un aperçu direct du rendu (ref.~\cite{starlogo}).

Là où StarLogo diffère est qu'il n'est pas nécessaire de connaître une seule ligne de code. En effet, StarLogo se base sur un principe de bouton. Pour une partie de code donnée, un bouton y correspond. Les boutons peuvent être liés entre eux, permettant de créer des actions plus complexes.
Tout bouton peut être positionné dans une page, qui correspond aux différents «~environnements~» du modèle~: le monde globale, un certain type de tortue, les patchs, etc.
Par exemple, pour créer 12 tortues \verb|Turtles| lors d'une initialisation globale du modèle il, faut se positionner sur la page «~setup~», y placer le bouton «~setup~» attaché du bouton «~create Turtles - num~», puis modifier le «~num~» en 12.

StarLogo est composé de deux fenêtres~:
\begin{itemize}
  \item la fenêtre code~: c'est ici que les boutons sont placés dans les différentes pages pour générer le code~;
  \item la fenêtre vue~: on y aperçoit le résultats du code généré, le rendu pouvant être en 2D comme en 3D.
\end{itemize}

Au niveau historique, StarLogo avait d'abord été crée pour Mac lors de la première version, puis après plusieurs années, une version pour tout type d'environnement a vu le jour et la version uniquement pour Mac fut rebaptisée MacStarLogo. Plusieurs versions sont apparues mais c'est la version 2.1 de 2004 qui reste la plus récente.

StarLogo est disponible sous la même licence et pour le même environnement d'exécution que son confrère NetLogo~: c'est un logiciel libre sous licence GPL qui fonctionne sur la machine virtuelle Java, d'où son gain de portabilité après la version MacStarLogo.
