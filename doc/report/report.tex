\documentclass[a4paper,11pt]{report}
\usepackage[T1]{fontenc}
\usepackage[utf8]{inputenc}
\usepackage{lmodern}
\usepackage[francais]{babel}
\usepackage{listings}
\usepackage[nottoc, notlof, notlot]{tocbibind}
\usepackage{rail}
\usepackage{graphicx}
\usepackage{listings}
\usepackage[babel=true]{csquotes}

\railalias{lbrace}{\{}
\railalias{rbrace}{\}}
\railalias{lpar}{(}
\railalias{rpar}{)}
\railalias{dot}{.}
\railalias{comma}{,}
\railalias{sharp}{\#}
\railalias{underscore}{\_}
\railalias{pipe}{|}
\railalias{ampersand}{\&}
\railalias{squote}{'}
\railalias{dquote}{"}
\railalias{tquote}{"""}
\railterm{lbrace,rbrace,lpar,rpar,dot,comma,sharp,underscore,pipe,ampersand,squote,dquote,tquote}

\title{Stibbons}
\author{Julia Bassoumi\\Florian Galinier\\Adrien Plazas\\Clément Simon}

\begin{document}

\maketitle
\tableofcontents

\begin{abstract}
Le langage Stibbons est un langage de programmation multi-agent pour programmeur débutant et avancé. Il est interprété en C++ et le rendu graphique est sous Qt. Il est accessible sous deux forme, la première donne accès à une interface graphique pour une visualisation du modèle généré. La seconde en ligne de commande pour une étude statistique des résultats obtenus.
Nous vous exposons ici notre organisation, ainsi que le fonctionnement du langage Stibbons.
\end{abstract}

\chapter{Introduction}
\section{Sujet}
	\subsection{Objectif}
	Le projet Stibbons est un projet visant à créer un interprète d'un dérivé de Logo, un langage de programmation permettant l'animation d'une «~tortue~» (cf \ref{Logo}). Nous avons choisi de développer un langage multi-agents, à l'instar de NetLogo ou StarLogo (cf \ref{NetLogo} et \ref{StarLogo}). Ainsi, l'objectif de ce projet était double~: la réalisation d'un interprète, capable d'analyser du code écrit dans un derivé de Logo, ainsi que la réalisation d'une application graphique capable de représenter l'évolution des agents.
	
	\subsection{Système multi-agents}
	Les systèmes multi-agents sont une approche de l'intelligence artificielle visant à facilité la résolution d'un problème en le découplant en plusieurs sous-problèmes. Ainsi, plutôt que de chercher à développer une intelligence unique complexe capable de résoudre le problème, l'approche multi-agent vise plutôt à créer des multitudes d'intelligences capables de résoudre qu'une petite partie du problème, et de compter sur l'intelligence collective émergeante pour voir apparaitre la solution au problème \cite{sma}.

	Ainsi, on peut prendre en exemple les fourmis qui, bien que n'ayant individuellement qu'une capacité «~limitée~», sont capables de survivre grâce à la synergie de leurs colonies.

\section{Cahier des charges}
	\subsection{Fonctionnalités attendues}
	Bien que la méthode de développement utilisée est fait apparaitre de nombreuses fonctionnalités souhaitables au fur à mesure du projet, un certain nombre d'entre elles nous ont semblés indispensables dès le début~:
	\begin{itemize}
		\item chaque agent devait évoluer parallèlement aux autres agents~;
		\item chaque agent devait être capable de communiquer avec les autres agents~;
		\item chaque agent devait pouvoir modifier d'une certaine façon le monde~;
		\item des structures des langages modernes de la programmation impératives devaient être présent dans le langage (tels que les conditionnelles, les boucles, les fonctions, etc.)~;
		\item l'utilisateur devait pouvoir voir le monde où évoluent ces agents.
	\end{itemize}


\chapter{Analyse de l'existant}
\section{Logo}
\label{Logo}

Le Logo est un langage apparu dans les années 60. Son objectif était de permettre à des personnes possédant peu de connaissances en informatique et en programmation (des enfants par exemple) de découvrir ce domaine de manière ludique et interactive. Ainsi, le langage permettait de diriger une tortue graphique, capable d'abaisser un crayon ou un feutre pour dessiner sur une feuille placé au sol. Les instructions entrées permettaient ainsi de tracer des formes, permettant une représentation très visuelle du code (la tortue physique est remplacé dans les implémentations moderne par une tortue virtuelle).

Le langage Logo en lui-même est un dérivé du Lisp (il est d'ailleurs parfois nommé Lisp sans parenthèses) et possède deux types de données~: les «~mots~» (chaîne de caractères) et les listes.

Du fait du public visé, les instructions de bases (par exemple \verb|forward|, \verb|left|, \verb|pendown|, etc.) et les structures du type procédures, boucles ou conditionnelles sont écrites de façon à être clairement explicite (cf. \ref{logo-proc}, \ref{logo-rpt} et \ref{logo-condi}).
Cependant, comme expliqué dans l'article \cite{Logo}, des études sur Logo ont montré que les enfants, hormis quelques exceptions, n'arrive pas à créer un programme entier et code «~ligne par ligne~» ce qui les empêchent de créer un modèle complexe et de cerner l'ensemble de la syntaxe de Logo.

\begin{lstlisting}[language=Stibbons,label=logo-proc,caption=Procédure en Logo]
to <nom de la procedure> :<parametre>
  <instructions>
  output <valeur a retourner>
end
\end{lstlisting}

\begin{lstlisting}[language=Stibbons,label=logo-rpt,caption=Boucle en Logo]
repeat <nb fois> [liste d'instruction]
\end{lstlisting}

\begin{lstlisting}[language=Stibbons,label=logo-condi,caption=Conditionnelles en Logo]
if <test> [liste d'instruction si vrai]
ifelse <test> [liste d'instruction si vrai] [liste d'instruction si faux]
\end{lstlisting}

Les instructions amènent la tortue à se déplacer, suivant une distance et un angle. On l'oriente ainsi suivant des coordonnées polaires.

\section{NetLogo}

NetLogo est un environnement de modélisation programmable pour simuler et observer des phénomènes naturels et sociaux au fil du temps. Il permet de donné des instructions et de regarder des agents réalisé ces dernières. On peut alors faire des observations et des connections inter-agents au niveau micro (individu par individu) comme au niveau macro (monde global).\\
NetLogo a été conçu pour des publics variés, notamment pour apprendre aux enfant à programmer, d'où sa syntaxe simple. Cependant, des études ont montré que les enfants, hormis quelques exceptions, n'arrive pas à créer un programme entier et code «~ligne par ligne~» ce qui les empêchent de créer un modèle complexe et de cerner l'ensemble de la syntaxe de NetLogo.
Il est aussi utilisé dans de nombreux domaines de recherche comme l'économie, la biologie, la physique,la chimie... et de nombreux articles ont été publié sur NetLogo.
\\

Au niveau du programme en lui-même, NetLogo est composé de 3 onglets~: 
\begin{itemize}
	\item onglet info~: c'est la documentation du code~;
	\item onglet code~: le code permettant de crée le modèle du monde ainsi que le comportement des agents y sont implémentés. On peut y retrouver les différentes procédures ainsi qu'une procédure «~Mere~» qui permettra d'initialisé le modèle. Certaines procédures, ou parfois attributs, peuvent être lié à des widgets dans l'interface~;
	\item onglet interface~: l'interface contient deux parties.
\end{itemize}


La partie «~observation~» qui est représenté par une fenêtre où l'on verra notre modèle dans le temps. Le rendu du modèle peut être en 2D comme en 3D.

La partie «~construction~» où l'on peut ajouter des widgets dans le but d’interagir avec le code. Par exemple un slider «~nb\_population~» qui permettra de choisir combien de tortue vont être crée sans modifier le code. On peut également y faire apparaître des boutons pour lancer on arrêter des procédures, des graphes pour observer des variations, des switch pour gérer les variables globales, des notes... On peut également contrôler le temps, ralentir pour mieux observer, accélérer pour voir ce que produit le modèle.
\\

NetLogo permet donc une interaction très rapide entre le code et le rendu graphique. Ceci permet aux développeurs de «~jouer~» avec leurs modèles en modifiant facilement certaines conditions et donc d'ajuster immédiatement leur code comme ils le souhaitent.

Une riche documentation et de nombreux tutoriels est fourni sur le site officiel du langage, ce qui permet une prise en main simple et ludique.

NetLogo est un logiciel libre et open source, sous licence GPL. Il fonction sur la machine virtuelle Java, et est donc opérable sur de nombreuses plate-formes (Mac, Windows, Linux, etc.).
D'après le site officiel, NetLogo est décrit comme la prochaine génération de langage de modélisation multi-agents, tout comme StarLogo.



\section{StarLogo}
\label{StarLogo}
Tout comme NetLogo, StarLogo est un environnement de modélisation programmable permettant de simuler et d'observer des phénomènes naturels et sociaux au fil du temps.
Ils ont les mêmes objectifs d'étude, d'éducation et de «~programmation facile~» ainsi qu'un aperçu direct du rendu (ref.~\cite{starlogo}).

Là où StarLogo diffère est qu'il n'est pas nécessaire de connaître une seule ligne de code. En effet, StarLogo se base sur un principe de bouton. Pour une partie de code donnée, un bouton y correspond. Les boutons peuvent être liés entre eux, permettant de créer des actions plus complexes.
Tout bouton peut être positionné dans une page, qui correspond aux différents «~environnements~» du modèle~: le monde globale, un certain type de tortue, les patchs, etc.
Par exemple, pour créer 12 tortues \verb|Turtles| lors d'une initialisation globale du modèle il, faut se positionner sur la page «~setup~», y placer le bouton «~setup~» attaché du bouton «~create Turtles - num~», puis modifier le «~num~» en 12.

StarLogo est composé de deux fenêtres~:
\begin{itemize}
  \item la fenêtre code~: c'est ici que les boutons sont placés dans les différentes pages pour générer le code~;
  \item la fenêtre vue~: on y aperçoit le résultats du code généré, le rendu pouvant être en 2D comme en 3D.
\end{itemize}

Au niveau historique, StarLogo avait d'abord été crée pour Mac lors de la première version, puis après plusieurs années, une version pour tout type d'environnement a vu le jour et la version uniquement pour Mac fut rebaptisée MacStarLogo. Plusieurs versions sont apparues mais c'est la version 2.1 de 2004 qui reste la plus récente.

StarLogo est disponible sous la même licence et pour le même environnement d'exécution que son confrère NetLogo~: c'est un logiciel libre sous licence GPL qui fonctionne sur la machine virtuelle Java, d'où son gain de portabilité après la version MacStarLogo.


\chapter{Analyse des outils}
\section{Outils de gestion de projet}

\subsection{Méthodes agiles}

Pour réaliser ce projet, nous avons choisi de faire de l'Agile, avec la méthode SCRUM. Nous nous sommes principalement servi des cours de gestion de projet de ce semestre, en particulier des interventions de Sandrine Maton. Faire de l'Agile permet d'avoir un rendu fonctionnel à chaque fin de sprint, et donc de s'assurer d'avoir un rendu final. De plus, c'est une manière de faire qui se dévellope en entreprise, nous voulions donc l'expérimenter.
\subsubsection{Backlog}
Nous avons commencer le projet avec un sprint 0, au cours duquel nous avons défini l'univers du projet, et ses fonctionnalités (backlog). Voir ~\ref{backlogv1} page~\pageref{backlogv1}.


\begin{figure}[h]
\caption{\label{backlogv1} Backlog version 0.1}
\includegraphics[scale=0.35]{doc/report/uml/backlogv1.png}
\end{figure}
Le backlog est composé de fonctionnalités, qui ont un indice entre 1000 et 0 pour leur importance du rendu final.

 On écrit aussi une \enquote{user-storie} pour chaque fonctionnalité. Cela correspond au test que l'on fait passer au programme pour valider la fonctionnalité.


A chaque sprint, on décide des fonctionnalités qu'on ajoute, et du nombre d'heures qu'on estime devoir y passer.


Voici pour comparer le backlog lors du début du sprint 4~\ref{backlogsp4} page~\pageref{backlogsp4} :


\begin{figure}[h]
\caption{\label{backlogsp4} Backlog sprint 4}
\includegraphics[scale=0.35]{doc/report/uml/backlogsp4.png}
\end{figure}
A la fin d'un sprint, on estime le temps passer sur chaque tâche, on fait le point sur notre avancée dans le projet.

Ici, on peut voir un ajout de fonctionnalités, le changement de statut des fonctionnalités réalisées, et le temps réel passer à les rendre utilisables.
Globalement, nous étions toujours dans les temps, car nous surestimions certaines tâches, qui s'averait plus facile que prévue.




\subsection{Git}
Le SIF de l'UM2 (ref.~\cite{GitLabSIF}) a mis a disposition des étudiants un service GitLab (ref.~\cite{GitLab}). Git est le gestionnaire de version associé à GitLab, c'est pourquoi nous l'avons choisi.
De plus, c'est un gestionnaire de version décentralisé, ce qui permet d'avoir accès à ceux-ci avec une simple connexion internet.
Il s'utilise avec des systèmes de branches et le plus souvent depuis un terminal, mais nous l'avons aussi utilisé avec Gitg, qui est une interface permettant d'avoir un visuel (cf.~\ref{gitg}).

\begin{figure}[h]
\caption{\label{gitg} Capture de Gitg}
\includegraphics[scale=0.35]{doc/report/uml/gitbranche.png}
\end{figure}

Les commandes les plus utilisées sont les suivantes~:
\begin{description}
\item[git checkout nom-branche] permet de changer de branche~;
\item[git branch] permet de savoir sur quelle branche on est~;
\item[git add fichiers] permet de suivre des fichiers (enregistrer les modifications qu'on fait sur des fichiers)~;
\item[git commit] permet de sauvegarder des changements des fichiers~;
\end{description}

De plus, GitLab possède une traqueur de bug. Ce dernier permet d'ouvrir des «~issues~» qui peuvent être fermées, ré-ouvertees, commentées, etc.


\subsection{Make}

Make est un utilitaire de construction de fichiers qui nous a été utile tout au long du projet pour construire les applications (stibbons et stibbons-cli), les tests unitaires, ou encore la documentation \LaTeX du projet. Il est également utilie à l'installation des applications.

Il fonctionne en laissant l'utilisateur définir des règles de construction de fichiers, en listant les fichiers dont sa construction dépend et en définissant les commandes nécessaires à sa construction.

Make est alors appelé à construire une cible (un fichier ou non), construisant son arbre de dépendances, vérifiant l'existance de ces dernières, les construisant ou reconstruisant au besoin. Ainsi Make permet d'éviter les compilations ou recompilations inutiles, accélerant et automatisant la construction de logiciels ou de documents.

\section{Tests unitaires}
\subsection{CppUnit}
CppUnit (ref.~\cite{CppUnit}) est un outil permettant d'organiser des tests unitaires.
On définit une classe de tests avec les attributs leurs étant nécessaires et des méthodes les réalisant. Cette classe est ensuite enregistrée dans le registre des tests pour être exécutée (cf.~\ref{TestAgent}).

CppUnit posséde des macros pour simplifier les tests comme~:
\begin{itemize}
\item \verb|CPPUNIT_ASSERT(test)|~;
\item \verb|CPPUNIT_ASSERT_EQUAL(v1,v2)|~;
\item etc.
\end{itemize}

Lors de l'exécution, il indique combien de tests ont réussi et échoué.
Son écriture est simple, et son organisation permet une lecture rapide des tests.


\section{C++11}

\subsection{GDB}


\section{Analyse de code}

\subsection{Flex}
\subsubsection{Théorie}
Pour faire la compilation, la première étape est l'analyse du fichier source.
Tout d'abord, on fait une analyse d'un point de vue lexicale, c'est à dire qu'on décompose les chaînes de caractère (le code) en lexème ou jeton.\\
L'une des façons de faire est de construire un automate à état fini associé au mots reconnus.\\
 Suite à la reconnaissance d'un mot ou lexème, l'analyseur lexicale retourne un jeton correspondant  à la catégorie lexicale du lexème. Plus précisément, on retourne un couple (jeton, valeur sémantique).\\
Par exemple, si on définit  (if , 300) et qu'on reconnaît un if on retourne (IF,).
On choisit plutôt des entiers pour représenter les catégories lexicales.\\
Pour les variables, il faut retourner une valeur sémantique, donc soit le lexème lui-même pour un littéral entier, l'indice d'entrée correspondant dans la table des symboles pour une variable.\\
Par exemple : (LITTERALCHAINE,'Bonjour !').\\
Pour éviter les erreurs avec les mots préfixes d'autres, on applique la règle du mot le plus long : on regarde le caractère suivant, s'il étend le lexème reconnu, on continue.\\
Il faut pouvoir revenir en arrière si on a été trop loin dans la lecture et qu'on se retrouve dans un état non terminal. Il faut donc connaître les états de notre automate.
Il faut aussi penser à filtrer les blancs et les commentaires, selon la grammaire.
Un générateur d'analyseur permet d'éviter cette étape.\\
\subsubsection{Pratique}
Flex est une version libre de l'analyseur lexical Lex. Il est généralement associé à l'analyseur syntaxique GNU Bison, la version GNU de Yacc. \\
Il lit les fichiers d'entrée donnés pour obtenir la description de l'analyseur à générer. La description est une liste de paires d'expressions rationnelles et de code C, appelées règles. \\
Flex a plusieurs régles pour l'écriture du fichier .l+ .\\
D'abord, un espace entre \%\{ \%\} qui contient une partie optionnelle de définition.\\
Par exemple :
\ \begin{verbatim}
%{
include <iostream.h>
class A\{ 
 public :
  void Hello() {cout<< ``Hello world !''<<endl ; }
};
%}
\end{verbatim}
Une partie obligatoire de régles lex commencant par \%\%.\\
Cette partie associe des instructions C++ à des expressions régulières.
\begin{verbatim}
%%
[a-z]([a-z])*           {return 5;}
\end{verbatim}
Enfin, une partie optionnelle pour des fonctions C++ définis par l'utilisateur, commençant par \%\%.\\
\begin{verbatim}
%%
int main(){
...
}
\end{verbatim}
Pour compiler : \textit{flex -+ exemple.l+}  puis \textit{g++ -g -o exemple lex.yy.cc -lfl}.\\
Après, c'est la fonction de l'analyseur syntaxique yyparse() qui appelle yylex() pour avoir les jetons correspondant au fichier lu.\\
La fonction main() appelle yyparse().\\


\subsection{Bison}

Yacc est un outil d'analyse syntaxique. L'analyse syntaxique permet de vérifier qu'un mot appartient bien au langage.\\ Il génère un analyseur syntaxique ascendant utilisant un automate à pile (dérivation à droite, on remplace le symbole non terminal le plus à droite).
 Son fonctionnement est le suivant : à chaque règle de grammaire, on associe des actions (instructions d'un langage). L'analyseur généré essaie de reconnaître un mot du langage défini par la grammaire. Il exécute les actions pour chaque règle reconnue.  Bison est une version de yacc.\\

\textbf{Exemple} :\\
D'après une grammaire ambiguë, on construit un vérificateur syntaxique. On écrit un source yacc : fichier.y, dans lequel on définit :

 \begin{verbatim}%{ \end{verbatim}  \textit{déclaration C}  \begin{verbatim}%} \end{verbatim}
 \begin{verbatim}%%\end{verbatim} \textit{définition de la grammaire reconnu} \begin{verbatim}%% \end{verbatim}
\textit{définition fonction C}\\

Les définitions de fonction C  doit avoir une fonction main, qui appelle yyparse(), une définition de yylex() appellée par yyparse() et une définition de yyerror(char*) pour signaler un erreur à l'utilisateur.
On compile avec bison : \textit{bison -y fichier.y} , puis \textit{gcc -o fichier.y.tab.c}, puis lancer l’exécutable.\\
Pour utiliser C++ avec bison, on écrit en C++ dans la partie action. On aura du C et C++ dans l'analyseur : fonction C, et action en C++.\\

Si on veut un analyseur syntaxique en C++, il faut utiliser un squelette de parseur C++ en utilisant soit l'option bison -skeleton=lalr1.cc ; soit en utilisant la directive \%skeleton « lalr1.cc ».\\ Ne pas oublier de déclarer yylex() aprés \%union !\\
\textit{Pour plus de détails, regarder le cours de Michel Meynard, dont ces informations sont principalement tirés.}

\section{Qt}

Qt est un framework d'application multiplateforme écrit en C++ principalement utilisé pour la création d'interfaces graphiques.

\subsection{Multi plateforme et multi langages}

Qt est utilisable sur de nombreuses plates-formes telles que Windows, Mac OS X, X11, Wayland, Android ou iOS.

De plus, bien que Qt soit développé en C++, ce n'est pas le seul langage depuis lequel il est utilisable, on trouve notament des bindings pour Python, JavaScript, Go, Ruby, Haskell ou encore Ada.

\subsection{Modules}

Qt comprend de nombreux modules afin d'aider tant que possible le développement d'applications.
On peut particulièrement citer :
\begin{description}
	\item[Core]~: une implémentation des types de base (QString, etc), conteneurs, parallèlisme, entrées-sorties, système d'évènements, etc~;
	\item[Widgets]~: des widgets pour le développement d'interfaces graphiques~;
	\item[Network]~: support de divers protocoles réseau (TCP, UDP, HTTP, SSL, etc)~;
	\item[Multimedia]~: lecture audio et vidéo~;
	\item[SQL]~: accès à des bases de données comprenant SQL~;
	\item[WebKit]~: moteur de rendu HTML~;
\end{description}

\subsection{Concepts fondamentaux}

\subsubsection{Widgets et layouts}

Qt propose un système de widgets complet et puissant. Il propose de nombreux widgets classiques tels que des boutons, des choix à puce, des onglets, des étiquettes, des images, etc.

Pour Qt, tout widget peut contenir des enfants et les arranger selon une disposition qui lui est affectée. Un widget n'ayant pas de widget parent sera considéré comme étant une fenêtre.
Son fonctionnement est ainsi assez différent de son concurrent Gtk+.
%pour qui seuls les widgets descendants de Container peuvent contenir des enfants, les dispositions étant des widgets conteneurs, et pour qui une fenêtre est un widget descendant de Window, classe descendant elle même de Container.

\subsubsection{Signaux et slots}

Qt propose également un système de signaux et de slots permettant d'implémenter le modèle observeur de manière efficace.% ?? %

Ainsi un widget peut émettre des signaux contenant ou non des données (par exemple, pour signaler le changement de valeur d'une entrée) et un autre widget peut réceptionner ce signal dans un de ses slots, l'exécutant alors. %Son concurrent GObject propose un système similaire mais pour lequel toute fonction ayant le bon prototype (potentiellement une lambda) peut être déclenchée par un signal, il n'y a ainsi donc pas de notion de slot.

\subsubsection{Une extension à C++}

Qt propose une extension à C++~: il y ajoute des mots-clés pour permettre de simplifier la définition d'objets descendants de QObject.
, tout particulièrement en spécifiant un ensemble de slots d'une certaine visibilité.
Ainsi lors de la déclaration d'une classe, il est possible de déclarer une liste de slots publics ou de signaux en les précédant des mentions \verb|public slots| et \verb|signals|, respectivement.
Le Meta-Object Compiler de Qt est alors utilisé pour convertir ces définitions en C++ classique à la compilation.

Qt propose également un système permettant d'embarquer des ressources (images, sons, ...) directement dans le binaire produit via la définition d'un fichier de collection de ressources (\verb|.qrc|) et l'utilisation d'un Ressource Compiler.

qmake est un générateur de Makefiles permettant de simplifier l'utilisation du Meta-Object Compiler et du Ressource Compiler.

\subsection{Outils}

Qt permet de développer des interfaces dans un langage déclaratif basé sur XML. Pour utiliser ce puissant atout, il est conseillé d'utiliser Qt Designer, une application permettant de dessiner des interfaces graphiques de manière intuitive, en plaçant manuellement les widgets les uns dans les autres.

Qt Creator va encore plus loin puisqu'il est un IDE assez complet centré sur le développement d'applications avec Qt. En effet il permet de gérer des projets, d'éditer du code, de designer des interfaces comme Qt Designer et même de créer des slots de manière graphique. Bien entendu il permet également de compiler, exécuter et débuguer le projet.

\subsection{Dessiner avec Qt}

Il est possible de réaliser des widgets personnalisés en redéfinissant la méthode \verb|paintEvent| d'un widget et en utilisant la classe QPainter pour dessiner sur le widget.

Une instance QPainter est liée à un widget et propose diverses méthodes permettant de dessiner~:
\begin{itemize}
	\item des lignes
	\item des arcs de cercle
	\item des polygones
	\item des ellipses
	\item des images
	\item du texte
\end{itemize}


\section{Latex}

\subsection{Généralités}
\LaTeX{} est un langage de rédaction de document qui force à avoir une structure sur la forme et le contenu. Il est notamment utilisé lors d'écritures de documents scientifiques car son écriture de contenus complexes (équations, bibliographie, etc.) se manie facilement.
Contrairement à d'autres logiciels de rédaction tel que LibreOffice, OpenOffice, etc \LaTeX{} n'est pas de type WYSIWYG (What You See Is What You Get). Il faut donc explicité la mise en page du document, d'où sa catégorie de langage.

\LaTeX{} est un logiciel libre.

\subsection{Beamer}
Beamer est un paquet spécialisé de LaTeX pour la création de présentation, souvent sous forme de diapositive. Plusieurs thèmes existent pour la mise en forme et, comme \LaTeX{}, Beamer n'est pas de type WYSIWYG.



\chapter{Modèle}
\section{Sprint 1}
Lors du premier sprint, nous avons dû mettre en place la base du programme.\\ Pour le modèle, il s'agissait de mettre en place les classes :
\begin{itemize}
\item Turtle qui représente une tortue, 
\item Point, pour stocker les coordonnées d'une tortue,
\item Line, qui permettent aux tortues de laisser une trace (instruction pendown),
\item World.\\
\end{itemize}
Voici l'UML correspondant :\\
\includegraphics[scale=0.5]{doc/report/uml/v01.png}
\newpage
La classe World représente le monde des tortues. Il contient la liste des tortues, et les lignes tracées par ces tortues.
Nous mettons aussi en place certains types Stibbons, comme Color, pour la couleur, ou Nil qui a une valeur nulle. Ils héritent de Value, une classe abstraite qui contient une valeur et ses accesseurs.\\

A la fin du premier sprint, nous avions réalisé ce schéma, en dehors de la classe Instruction qui n'était pas nécessaire. Les instructions tel que pendown, forward, etc... sont finalement des mots clés dans la grammaire du langage.\\
\includegraphics[scale=0.5]{doc/report/uml/v01reel.png}
CAPTURE D'ECRAN DE L'APPLI

\section{Sprint 2}
Pour le second sprint, nous avons ajouté une classe Agent, car les classes World, Turtle et Zone sont des agents, au sens multi-agent.\\ Ils ont chacun un parent, celui qui l'a créer, une liste d'enfant, et des propriétés. Les propriétés sont des variables définit par l'utilisateur lors de la définition du code de l'agent, donc surtout utile pour les tortues.\\
Le monde a une taille, une liste de "breed", d'espèce. Il y a les espèces nommées et les "anonymes", qui sont des turtles dans le code.\\
Comme le montre ce code, on peut créer des tortues nommées ou pas.\\
METTRE UN EXEMPLE DE CODE DE L'UTILISATION DE BREED\\

Les types stibbons sont les mêmes, mais leurs définitions s'est un peu compléxifier en passant par une classe Simple-value, pour la mise en place des mutexs. Une énumeration des types Stibbons existe, elle est utiliser avec la méthode getType() pour pouvoir connaître le type de Value.\\
Pour que l'utilisateur puisse écrire des fonctions dans le code, nous avons ajouté une classe Function, qui stocke un arbre, qui contient le code la fonction, déja parser.\\
Le plus important, nous avons mis en place les threads dans l'interpreteur. Un nouveau thread correspond à une nouvelle tortue, c'est à dire, par rapport au code, à un "new agent".\\ Des mutexs ont été ajouté dans toutes les classes pour assurer que les threads sont thread-safe.\\
A la fin de ce sprint, nous devons relancer le programme pour charger un nouveau fichier de code, ce qui n'est pas pratique.\\
\includegraphics[scale=0.45]{doc/report/uml/v02reel.png}
CAPTURE D'ECRAN DE L'APPLI
\newpage

\section{Sprint 3}

Lors du sprint 3, nous avons mis en place des pointers intelligents dans toutes nos classes REFERENCE A UN ARTICLE.\\
Nous avons ajouté des sous-classes de Function Standart-function et User-function pour représenter les fonctions standart du langages. On les différencies des instructions par leurs parenthéses derrière leur nom.\\ Les user-function sont ...A COMPLETER.\\
L'ajout de table fait aussi partit de ce sprint. Nous avons choisi, pour notre langage un seul type de conteneur : les tableaux.
Nous avons choisi de les écrire façon php, avec des accolades.\\
EXEMPLE DE CODE !!\\
Le but de ce sprint été la mise en place de la communication entre les agents. Les tortues peuvent "communiquer" avec les zones par écriture dans leurs propriétés. Les tortues peuvent communiquer entre elles grâce à des Standart-function comme sendAll(), recv(), etc...\\
\includegraphics[scale=0.4]{doc/report/uml/v03.png}

\section{Sprint 4}
Nous devons ajouter des fonctionnalités tel que la maitrise du temps et l'exportation du modèle. Ces ajouts ne provoque pas de changement du coté du modèle, si ce n'est quelques méthodes dans les classes World et Turtle et zone pour l'exportation du modèle.\\
L'exportation du modèle consiste à créer une sauvegarde de l'état du modèle à un instant t dans un fichier JSON.\\
Cela permettra ensuite, en passant par une transformation en CSV, d'avoir un tableau avec toutes ces données, ce qui permettra d'avoir des diagrammes de l'évolution du monde.\\


\chapter{Interprète}
L'application Stibbons fournit un interprète (ou interpréteur) pour le langage du même nom. Cette interprétation du code se déroule en deux phases~: une de compilation lors du chargement du code, et une d'interprétation de l'arbre abstrait (généré lors de la première phase) lors de l'exécution.

La première phase peut elle-même être découpée en deux parties~:
\begin{itemize}
\item l'analyseur lexical qui «~lit le flot de caractères qui constituent le programme source et les regroupe en séquences de caractères significatives appelées \emph{lexèmes}.~» \cite{compilateurs}~;
\item l'analyseur syntaxique qui, à partir des lexèmes, génère un arbre abstrait qui pourra par la suite être analysé pour être interprété.
\end{itemize}

L'interprétation quant à elle se déroule lors de l'analyse sémantique de l'arbre abstrait, qui exécute les opérations contenues dans les nœuds.

\section{Analyseur lexical}
\label{analyse-lexicale}
L'analyse lexicale vise à produire un flot de jeton qui pourront être analyser par l'analyseur syntaxique. Ces jetons sont des paires composées d'un type de jeton et de la valeur du lexème (par exemple, l'analyse du lexème \verb|12| va générer le jeton \verb|<NUMBER,12>| dans notre cas). Certains lexèmes peuvent générer des jetons qui n'ont pas de valeur (par exemple, le lexème \verb|(| va entrainer la génération du jeton \verb|<(>|).

Nous avons fait le choix d'utiliser l'outil Flex pour notre projet (cf \ref{Flex}).

\subsection{Jeton}


\subsection{Règles}


L'analyseur lexical de Stibbons est écrit en Flex et permet de détecter certaines chaînes de caractères pour y générer des séquences correspondantes dans l'analyseur syntaxique.

Par exemple, lors de la détection d'un mot (prenons le mot \verb|recv|), la séquence correspondante est générer (ici, la séquence RECV serait produite).
Un comptage de ligne pour la gestion d'erreur y est également faite.

%%Et là en dessous c'est un copié collé de la partie de julia

\subsubsection{Théorie}
Pour faire la compilation, la première étape est l'analyse du fichier source.
Tout d'abord, on fait une analyse d'un point de vue lexicale, c'est à dire qu'on décompose les chaînes de caractère (le code) en lexème ou jeton.\\
L'une des façons de faire est de construire un automate à état fini associé au mots reconnus.\\
Suite à la reconnaissance d'un mot ou lexème, l'analyseur lexicale retourne un jeton correspondant  à la catégorie lexicale du lexème. Plus précisément, on retourne un couple (jeton, valeur sémantique).\\
Par exemple, si on définit  (if , 300) et qu'on reconnaît un if on retourne (IF,).
On choisit plutôt des entiers pour représenter les catégories lexicales.\\
Pour les variables, il faut retourner une valeur sémantique, donc soit le lexème lui-même pour un littéral entier, l'indice d'entrée correspondant dans la table des symboles pour une variable.\\
Par exemple : (LITTERALCHAINE,'Bonjour !').\\
Pour éviter les erreurs avec les mots préfixes d'autres, on applique la règle du mot le plus long : on regarde le caractère suivant, s'il étend le lexème reconnu, on continue.\\
Il faut pouvoir revenir en arrière si on a été trop loin dans la lecture et qu'on se retrouve dans un état non terminal. Il faut donc connaître les états de notre automate.
Il faut aussi penser à filtrer les blancs et les commentaires, selon la grammaire.
Un générateur d'analyseur permet d'éviter cette étape.\\



\section{Analyseur syntaxique}
\label{analyse-syntaxique}
L'analyse syntaxique permet de vérifier que la structure d'un programme est bien en accord avec les règles de grammaire du langage. Par exemple, la grammaire de notre langage comporte une règle qui indique que le jeton \verb|<FD>| doit être suivi d'un jeton \verb|<NUMBER,>|. Le but de notre analyse ici est double~: vérifier que le programme est bien un programme de notre langage valide, mais également généré un arbre abstrait qui pourra facilement être analysé par notre analyseur sémantique.

Nous avons fait le choix d'utiliser GNU Bison (cf.~\ref{bison}) dans notre projet.

\subsection{Arbre abstrait}
Un arbre abstrait est une structure d'arbre dont chaque noeud feuille représente les opérandes des opérations contenues sur les autres noeuds. Cet arbre peut être considéré comme une forme de code intermédiaire, et peut être interprété très facilement en interprétant pour chaque noeud, les sous-arbres et en appliquant l'opération du noeud courant.

\begin{lstlisting}[language=Stibbons,label=arbre-code,caption=Exemple de code Stibbons]
  agent turtle (a) {
    teleport(50,50,0)
    fd a
  }

  b = 18
  repeat b {
    if(b == 18) {
      new turtle(b)
    }
    else {
      new turtle(3)
    }
    b = b - 1
  }
\end{lstlisting}

\begin{figure}[h]
\centering
\includegraphics[scale=0.8]{doc/report/img/arbre-abstrait}
\caption{\label{arbre-abstrait} Arbre abstrait généré lors de l'analyse du code \ref{arbre-code}}
\end{figure}

L'avantage de générer un tel arbre est qu'il est bien plus rapide d'effectuer un parcours d'arbre à chaque interprétation plutôt que de refaire une analyse syntaxique à chaque fois.

\subsection{Fonctionnement}

\subsubsection{Analyse syntaxique LALR}


\section{Analyseur sémantique}
\label{Analyseur sémantique}
L'analyse sémantique du Stibbons est effectué dans les classes Interpreter. Plus exactement dans 3 classes. Voir l'UML~\ref{interpreterUML} correspondant page~\pageref{interpreterUML}.

\begin{figure}[h]
\caption{\label{interpreterUML} UML de l'analyseur sémantique}
\includegraphics[scale=0.5]{doc/report/uml/interpreterUML.png}
\end{figure}

On a donc les classes \verb|TurtleInterpreter| et \verb|WorldInterpreter| qui héritent de la classe Interpreter. Cette dernière analyse et provoque, dans le modèle, les actions liées au code stibbons écrit. Elle permet de gérer tout type d'interaction, du moment que ces actions ne sont pas liées à un monde ou une tortue. La classe \verb|TurtleInterpreter|, gère justement ce dernier type d'actions : celles liées à une tortue. Contrairement à \verb|WorldInterpreter| qui gère les actions du monde.

L'utilité de la classe \verb|InterpreterManager| est expliqué de manière détaillée dans la section \ref{remaniementInterpreter} (page~\pageref{remaniementInterpreter}).

Ainsi, pour avoir un exemple concret, si on demande à une tortue d'avancer en stibbons (ex : \verb|fd 10|), alors c'est le \verb|TurtleInterpreter| qui gèrera cette action.
A contrario, si on effectue une définition de fonction (ex : \verb|a = 10|) alors l'\verb|Interpreter| affectera la valeur \verb|10| à la propriété \verb|a| de l'agent courant.
Pour ce qui est du WorldInterpreter, il n'y a pas encore d'exemple possible, tout simplement car pour l'instant il n'y a pas d'actions spéciales pour le monde.

\subsection{Fonctionnalités}

\subsubsection{Sprint 1 \& 2}
Les deux premiers sprints ont été assez conséquent au niveau du nombre de fonctionnalités ajoutées.
En effet lors du sprint 1 on pouvait déjà effectuer les opérations de bases sur une tortue, telles que avancer, tourner, écrire, etc. De plus les opérations de calculs ainsi que la gestions de nombre ont été fait. En effet, il était nécessaire de gérer les nombres de manière à pouvoir indiquer à la tortue de quelle distance elle devait avancer.

Lors du deuxième sprint sont apparu les conditionnelles, ainsi que les boucles, les comparaisons binaires, les booléens et autres type (color, string, etc.), la création de nouveaux agents dans le code et les fonctions sans paramètres. Ce sprint fut alors une version déjà bien avancé de notre programme final.


\subsubsection{Sprint 3 à 5}
Les sprints suivants ont été plus léger. Non pas parce qu'il y avait moins de travail, car la gestion de l'interpréteur fut remanier à ce moment là, mais car il y avait moins de fonctionnalités à ajouter. En effet à partir du sprint 3, les fonctionnalités suivantes ont été rajoutés~:
\begin{itemize}
\item accès à la parenté et aux zones (en lecture seulement)~;
\item ajout des tables et de boucles dédiés (\verb|foreach|)~;
\item gestion de la vitesse et de la pause~;
\item ajout des communications avec le \verb|send| et le \verb|recv|.
\end{itemize}


\subsection{Remaniement de l'analyseur sémantique}
\label{remaniementInterpreter}

L'analyseur sémantique a d'abord été une seule classe~: la classe \verb|Interpreter|.
Cette dernière implémentait tout types d'actions a effectué pour n'importe quel type d'agent.
Cependant, lors de l'arrivé de la fonctionnalité d'ajout d'un nouvel agent (\verb|new agent|), nous nous sommes aperçu qu'il serait mieux d'avoir un interpréteur par type d'agent, ou plus précisement un interpréteur pour le monde, un pour les tortues et un pour les actions communes aux deux types.
Nous avons alors crée les deux classes~: \verb|WorldInterpreter| et \verb|TurtleInterpreter|.

De manière parallèle, la création du monde s'effectuait dans l'\verb|Interpreter|, puis dans le \verb|WorldInterpreter|, il fut alors nécessaire de crée une classe qui gérerait à la fois la création du monde et tout ce qui concernait l'application (la pause, le temps, les interpreteurs eux-mêmes). Nous avons alors décidé de crée la classe \verb|InterpreterManager|.
Cette classe connait ainsi tout les interpréteurs, permet de les prévenir d'une éventuelle pause du programme, de stocker les threads correspondants à chaque interpréteur, de crée un monde avec les pré-directives choisies~; c'est une sorte de gestionnaire d'interpréteurs.



\chapter{Interface graphique}
\input{doc/report/interface.tex}

\chapter{Conclusion}
Durant quatre mois, nous avons réalisé ce projet que nous avions imaginé.
Tout d'abord, ce projet nous a permis de pratiquer la méthode Agile, ce qui était une bonne expérience et, avec le recul, un bon choix, car nous avions un projet fonctionnel toutes les deux semaines.

Nous avons su gérer le coté compilation et le coté programmation C++ de ce TER, avec l'aide de notre tuteur.
Nous avons imaginé au milieu du projet l'application sans interface, complétaire à notre programme principale.
Nous avons découvert de nouveaux outils tel que Qt ou appronfondi des connaissances, comme en C++, ou en flex.
 C'est avec plaisir que nous avons fini la version 1.0 pour la soutenance.

Il y a cependant des choses à améliorer, comme notre abscence de pile dans le programme. Cela limite la récursion du code.
L'écriture de la sortie standart pourrait être afficher dans le programme. Nous voulions ajouter des entrés sur le programme, comme un slider pour faire varier le nombre de tortue.Et éventuellement, l'ajout de bulle pour afficher les messages reçu par chaque tortue serait un plus.

Nous retiendrons néanmoins que nous avons réalisé notre objectif : avoir un interprêteur de notre langage, le Stibbons et une interface visuel du code qui s'exécute.

\appendix

\chapter{Documentation}

\section{Syntaxe}

\subsection{Flex}

\begin{rail}
	ID : ( 'a-z' | underscore ) ( ( 'a-z'  | underscore | '0-9' ) * ) ;
	FD : 'fd' | 'forward'	;
	LT : 'lt' | 'turn-left'	;
	RT : 'rt' | 'turn-right' ;
	PD : 'pd' | 'pen-down' ;
	PU : 'pu' | 'pen-up' ;
	SEND : 'send'	;
	RECV : 'recv'	;
	DIE : 'die'	;
	AND : 'and' | ampersand ;
	OR : 'or' | pipe ;
	XOR : 'xor' | '\^{}' ;
	NOT : 'not' | '!'	;
	RPT : 'repeat' ;
	WHL : 'while' ;
	IF : 'if' ;
	ELSE : 'else' ;
	NEW : 'new' ;
	AGT : 'agent' ;
	FCT : 'function' ;
	FOR : 'for' ;
	IN : ':' | 'in' ;
	TYPE : nullt
	| numbert 
	| booleant 
	| stringt 
	| colort
	| tablet
	| typet
	| turtlet
	| zonet 
	| worldt
	;
	NIL : 'null' ;
	BOOLEAN : 'true' | 'false' ;
	STRING : squote ( WORD * ) squote
	| dquote ( WORD * ) dquote
	| tquote ( WORD * ) tquote
	;
	NUMBER : (('0-9' + ) ( (dot ( '0-9' * ) ) ?))
	| dot ( '0-9' + ) ;
	COLOR : sharp ('a-f' | '0-9') ('a-f' | '0-9') ('a-f' | '0-9') \\
	( ('a-f' | '0-9') ('a-f' | '0-9') ('a-f' | '0-9') ) ? ;
\end{rail}


\subsection{Bison}

\begin{rail}
code : worlddirlist statementlist | statementlist ;

worlddirlist : worlddir
| worlddirlist worlddir
;

worlddir : '\%' ID lit 'newline' 
| 'newline'
;

statementlist : ( statement | statementlist statement ) ?
;

statement : exprstatement | complstatement;	

complstatement : bloc
| declstatement 
| selection 
| loop 
| complstatement 'newline'
;

bloc : lbrace rbrace 
| lbrace statementlistbloc rbrace 
; 

statementlistbloc : statementlist 
| statementlist exprnoseparator 
;

declstatement : ( AGT | FCT ) ID lpar ( ( ID ( (',' ID) *) ) ? ) rpar bloc 
;

selection : IF expr statement (ELSE statement) ? ;

loop : RPT expr statement
| WHL expr statement
| FOR (( lpar ( ID | ID ',' ID ) IN expr rpar ) | ((ID | ID ',' ID ) IN expr)) statement 
;

exprstatement : 'newline' 
| expr 'newline'
| instrturtle 'newline' 
| creatstatement 'newline'
;

exprnoseparator : expr 
| instrturtle
| creatstatement 
;

expr : assignmentexpression 
| lpar expr rpar
| expr ( '+' 
| '-' 
| '/' 
| '*' 
| '\%' 
| AND 
| OR 
| XOR 
| '==' 
| '!='
| '>'
| '>=' 
| '<'
| '<=' ) expr
| '-' expr 
| NOT expr
| primaryexpr
| lit 
;

assignmentexpression : primaryexpr '=' expr
| primaryexpr '=' lbrace \\ (
(expr ((',' expr ) *))
| (expr ':' expr ((',' expr ':' expr ) *))
)? rbrace
;

primaryexpr : ID
| primaryexpr '.' ID
| primaryexpr '[' expr ']'
| primaryexpr '[' ']'
| ID lpar ( ( expr ((',' expr) *) ) ? )  rpar
;

creatstatement : (primaryexpr '=') ? \\ (NUMBER | primaryexpr)? (NEW AGT bloc
| NEW ID lpar \\ ( ( expr ((',' expr) *) ) ? ) rpar)
;

instrturtle : FD expr
| LT expr
| RT expr
| PU
| PD
| SEND expr expr
| SEND expr
| RECV primaryexpr primaryexpr
| RECV primaryexpr
| DIE ;

lit : NUMBER 
| STRING 
| BOOLEAN 
| COLOR 
| NIL
| TYPE ;
\end{rail}



\section{Propriétés standard}

Les agents possèdent certaines propriétés par défaut. Ces propriétés varient selon leur type, et peuvent être de simples valeurs en lecture seule, des fonctions en lecture seule, ou encore des propriétés spéciales ayant une sémantique particulière et requérant un type précis.

\subsection{Attributs communs à tous les agents}

\begin{description}
	\item[black] $\rightarrow$ \#000000

	Le noir.

	\emph{Attribut en lecture seule.}

	\item[white] $\rightarrow$ \#ffffff

	Le blanc.

	\emph{Attribut en lecture seule.}

	\item[grey] $\rightarrow$ \#7f7f7f

	Un gris moyen.

	\emph{Attribut en lecture seule.}

	\item[red] $\rightarrow$ \#ff0000

	Le rouge.

	\emph{Attribut en lecture seule.}

	\item[green] $\rightarrow$ \#00ff00

	Le vert.

	\emph{Attribut en lecture seule.}

	\item[blue] $\rightarrow$ \#0000ff

	Le bleu.

	\emph{Attribut en lecture seule.}

	\item[yellow] $\rightarrow$ \#ffff00

	Le jaune.

	\emph{Attribut en lecture seule.}

	\item[cyan] $\rightarrow$ \#00ffff

	Le cyan.

	\emph{Attribut en lecture seule.}

	\item[magenta] $\rightarrow$ \#ff00ff

	Le magenta.

	\emph{Attribut en lecture seule.}

\end{description}

\subsection{Fonctions communes à tous les agents}

\begin{description}
	% print
	\item[print] (value) $\rightarrow$ null

	Imprime une valeur sur la sortie standard.

	\begin{description}
		\item[value] La valeur à imprimer
		\item[Retourne] null
	\end{description}

	% println
	\item[println] (value) $\rightarrow$ null

	Imprime une valeur dans une nouvelle ligne sur la sortie standard.

	\begin{description}
		\item[value] La valeur à imprimer
		\item[Retourne] null
	\end{description}

	% rand
	\item[rand] () $\rightarrow$ number

	Retourne un entier positif au hasard.

	\begin{description}
		\item[Retourne] un entier positif au hazard
	\end{description}

	% random
	\item[random] (min, max) $\rightarrow$ number

	Retourne un entier positif au hasard compris entre les bornes indiquées.

	\begin{description}
		\item[min] La valeur borne minimale inclusive
		\item[max] La valeur borne maximale exclusive
		\item[Retourne] un entier positif au hazard compris entre les bornes indiquées
	\end{description}

	% type_of
	\item[type\_of] (value) $\rightarrow$ type

	Retourne le type d'une valeur.

	\begin{description}
		\item[value] La valeur dont on veut obtenir le type
		\item[Retourne] le type de la valeur
	\end{description}

	% type_of
	\item[size] (table) $\rightarrow$ number

	Retourne le nombre d'éléments contenus dans une table.

	\begin{description}
		\item[table] La table dont on veut connaître le nombre d'éléments
		\item[Retourne] le nombre d'éléments contenus dans la table
	\end{description}
\end{description}

\subsection{Attributs du monde}

\begin{description}
	\item[max\_x] $\rightarrow$ number

	L'abscisse maximale du monde.

	\emph{Attribut en lecture seule.}

	\item[max\_y] $\rightarrow$ number

	L'ordonnée maximale du monde.

	\emph{Attribut en lecture seule.}
\end{description}

\subsection{Fonctions du monde}

\begin{description}
	% ask_zones
	\item[ask\_zones] (function) $\rightarrow$ null

	Exécute une fonction sur chaque zone.

	\begin{description}
		\item[function] La fonction à exécuter sur chaque zone
		\item[Retourne] null
	\end{description}

\end{description}

\subsection{Attributs des tortues}

\begin{description}
	\item[color] $\rightarrow$ \#000000

	La couleur de la tortue.

	\emph{Requiert une valeur de type color.}

	\item[parent] $\rightarrow$ agent

	L'agent qui a créé la tortue.

	\emph{Attribut en lecture seule.}

	\item[pos\_x] $\rightarrow$ number

	L'abscisse de la position de la tortue.

	\emph{Requiert une valeur de type number.}

	\item[pos\_y] $\rightarrow$ number

	L'ordonnée de la position de la tortue.

	\emph{Requiert une valeur de type number.}

	\item[pos\_angle] $\rightarrow$ number

	L'angle de la tortue.

	\emph{Requiert une valeur de type number.}

	\item[world] $\rightarrow$ world

	Le monde.

	\emph{Attribut en lecture seule.}

	\item[zone] $\rightarrow$ zone

	La zone dans laquelle est la tortue.

	\emph{Attribut en lecture seule.}
\end{description}

\subsection{Fonctions des tortues}

\begin{description}
	% distance_to
	\item[distance\_to] (turtle) $\rightarrow$ number

	Retourne la distance la plus courte vers une autre tortue.

	\begin{description}
		\item[turtle] La tortue vers laquelle obtenir la distance
		\item[Retourne] null
	\end{description}

	% face
	\item[face] (turtle) $\rightarrow$ null

	Tourne la tortue pour qu'elle fasse face à une autre par le chemin le plus court.

	\begin{description}
		\item[turtle] La tortue à laquelle faire face
		\item[Retourne] null
	\end{description}

	% in_radius
	\item[in\_radius] (distance) $\rightarrow$ table

	Retourne l' ensemble de tortues dans le rayon donné autour de la tortue.

	\begin{description}
		\item[distance] Le rayon autour de la tortue à sonder
		\item[Retourne] Une table contenant les tortues dans le rayon
	\end{description}

	% inbox
	\item[inbox] () $\rightarrow$ number

	Retourne le nombre de message non lus.

	\begin{description}
		\item[Retourne] Le nombre de messages non lus
	\end{description}

	% teleport
	\item[teleport] (x, y, angle) $\rightarrow$ null

	Téléporte une tortue à une certaine coordonnée et avec un certain angle.

	\begin{description}
		\item[x] L'abscisse où téléporter la tortue
		\item[y] L'ordonnée où téléporter la tortue
		\item[angle] L'angle à donner à la tortue
		\item[Retourne] null
	\end{description}
\end{description}

\subsection{Attributs des zones}

\begin{description}
	\item[color] $\rightarrow$ \#ffffff

	La couleur de la zone.

	\emph{Requiert une valeur de type color.}

	\item[parent] $\rightarrow$ agent

	L'agent qui a créé la zone.

	\emph{Attribut en lecture seule.}

	\item[world] $\rightarrow$ world

	Le monde.

	\emph{Attribut en lecture seule.}
\end{description}



\chapter{Tutoriel}

\section{Tutoriel}

\subsection{Salut, monde !}

Commençons par imprimer du texte.

\begin{verbatim}
println("Salut, monde !")
\end{verbatim}

Ici l'agent par défaut, le monde, appelle la fonction println avec pour paramètre la chaîne de caractères "Salut, monde !", ce qui a pour effet d'imprimer ce texte dans une nouvelle ligne sur la sortie standard.

\subsection{Les premiers agents}

Créons maintenant des agents.

\begin{verbatim}
new agent {
    println("Salut, humain !")
}
\end{verbatim}

Le monde crée un nouvel agent mobile, une tortue, qui apparaîtra alors dans le monde et exécutera le code passé entre accolades.

Il est possible de créer plusieur tortues exécutant le même code en spécifiant leur nombre.

\begin{verbatim}
5 new agent {
    println("Salut, humain !")
}
\end{verbatim}

Ainsi, cinq tortues sont créés et chacune d'elles imprime "Salut, humain !".

\subsection{Dessiner un carré}

Les tortues peuvent se déplacer sur le monde en avançant et en tournant à gauche ou à droite. Elles ont également un stylo qu'elles peuvent abaisser ou relever afin de tracer des lignes sur le monde.

\begin{verbatim}
new agent {
    pd
    fd 50
    rt 90
    fd 50
    rt 90
    fd 50
    rt 90
    fd 50
    println("Voici un beau carré !")
}
\end{verbatim}

Ici, pd demande à la tortue d'abaisser son stylo (pen down), fd demande à la tortue d'avancer (forward) d'une certaine distance, et rt demande à la tortue de tourner d'un certain nombre de degrés.

\subsection{Répéter}

Afin d'éviter de se répéter, on peut demander à l'interprète de le faire un certain nombre de fois pour nous.

\begin{verbatim}
new agent {
    pd
    repeat 4 {
        fd 50
        rt 90
    }
    println("Voici qui est mieux. =)")
}
\end{verbatim}

\subsection{Boucler}

Il est également possible de boucler tant qu'une condition est vraie.

\begin{verbatim}
new agent {
    println("Je vais faire ma ronde.")
    while true {
        fd 50
        rt 90
    }
}
\end{verbatim}

\subsection{Agents typés}

Il est possible de définir un type d'agent sans en créer, afin d'en créer plus tard.

\begin{verbatim}
agent personne (nom) {
    println("Je m'appelle " + nom + ".")
}

new personne("Mathieu")
new personne("Michel")
\end{verbatim}

Ici, le type d'agent personne a été définit. Un type d'agent peut prendre des paramètres exactement de la même manière qu'une fonction.

Ainsi on a pu créer deux tortues de type personne, chacune ayant son propre nom.

\subsection{Fonctions}

Il est possible de définir des fontions. Les fonctions sont définies dans l'espace de nom des propriétés de l'agent.

\begin{verbatim}
agent fourmi () {
    function gigoter () {
        rt rand() % 60
        lt rand() % 60
        fd 1
    }

    while true {
        gigoter()
    }
}

new fourmi()
\end{verbatim}

On définit ici la fonction gigoter pour les agents de type fourmi, qui est utilisée un peu plus bas dans le code.

\subsection{Couleurs}

Les tortues ont une couleur qui peut être modifiée. C'est également cette couleur qu'elles utilisent pour dessiner sur le monde.

\begin{verbatim}
agent fourmi (couleur) {
    function gigoter () {
        rt rand() % 60
        lt rand() % 60
        fd 1
    }

    color = couleur
    pd

    while true {
        gigoter()
    }
}

new fourmi(red)
new fourmi(blue)
\end{verbatim}

\subsection{Zones}

Le monde est constitué de zones, qui sont eux aussi des agents. Les zones ont tout comme les tortues une couleur qui peut être modifiée.

\begin{verbatim}
agent fourmi (couleur) {
    function gigoter () {
        rt rand() % 60
        lt rand() % 60
        fd 1
    }

    color = couleur
    pd

    while true {
        gigoter()
        zone.color = couleur
    }
}

new fourmi(red)
new fourmi(blue)
\end{verbatim}

Ici, les fourmis changent la couleur des zones sur lesquelles elles passent.

\subsection{Messages}

Les tortues peuvent s'envoyer des messages. Les messages peuvent être envoyés à un destinataire précis ou à toutes les tortues.

\begin{verbatim}
destinataire = new agent {
    message = recv()

    println("Quelqu'un m'a dit : " + message)
}

new agent {
    send (destinataire, "Coucou !")
}
\end{verbatim}

\begin{verbatim}
5 new agent {
    message = recv()

    println("Quelqu'un m'a dit : " + message)
}

new agent {
    send-all ("Oyez, agents !")
}
\end{verbatim}

S'il n'y a aucun message dans sa boîte à messages, une tortue demandant la lecture d'un message sera bloquée jusqu'à réception d'un message à lire. Pour éviter un bloquage, il est possible de vérifier le nombre de messages présents dans la boîte.

\begin{verbatim}
new agent {
    new agent {
        send (parent, "Bonjour, parent !")
    }

    while true {
        if (inbox() > 0) {
            message = recv()

            println("Quelqu'un m'a dit : " + message)
        }

        rt 1
        fd 1
    }
}
\end{verbatim}



\chapter{Résumés des réunions}

\input{doc/report/reunion1}
\section{3 février 2015}
Lors de cette réunion, nous avons principalement discuté du projet, et nous nous sommes mis d'accord sur quelques points.\\
Tout d'abord, nous utiliserons un thread par tortue, car cela ne devrait pas poser de problème de performance, et cela permettra l'exécution en parallèle des tortues, ce qui change de NetLogo.\\ Nous avons parlé de mettre une option pour exécuter pas à pas ces threads.\\
A propos du vocabulaire du projet, nous avons choisi d'appeler les patchs de NetLogo des zones, et pour la tortue d'origine nous la nommons le dieu-tortue.\\
Pour la communication, elle sera autorisée entre tortues et zones, et éventuellement limitée par la distance.\\
Pour la partie technique, le main devrait gérer les zones, et la communication se fera avec des files de messages. Nous avons également parler d'utiliser des classes anonymes pour remplacer la fonction go dans les tortues.\\
C'est ce qui est ressorti de nos discussions, mais ces décisions ne sont pas encore définitive.
\section{10 février 2015}

Voici la liste des tâches à faire pour la semaine prochaine (si assignées).

\subsection{Analyse de l'existant}
\begin{itemize}
	\item Logo $\rightarrow$ Florian
	\item NetLogo $\rightarrow$ Clément
	\item StarLogo $\rightarrow$ Clément
\end{itemize}

\subsection{Analyse des outils}
\begin{itemize}
	\item Gestion de projet~:
	\begin{itemize}
		\item producteev
		\item openproj
		\item git
		\item make
	\end{itemize}
	\item Tests unitaires $\rightarrow$ Florian
	\item C++11 ou supérieur
	\begin{itemize}
		\item Threads standard C++ $\rightarrow$ Florian
		\item gdb
	\end{itemize}
	\item Flex, Bison $\rightarrow$ Julia
	\item Qt $\rightarrow$ Adrien
	\begin{itemize}
		\item Qt en général
		\item Dessiner avec Qt
	\end{itemize}
	\item LaTeX (et Beamer)
\end{itemize}


\section{24 février 2015}

Lors de cette réunion nous avons tout d'abord fait un point sur le backlog et les taches que nous avions choisi pour le premier sprint~:
\begin{itemize}
\item interprète simple, qui ne comprend que des instructions basiques~;
\item une seule tortue est gérée~;
\item pas d'éditeur intégré dans un premier temps mais seulement lecture de code dans un fichier externe.
\end{itemize}
Nous avons également précisé que le code analysé serait lu en intégralité et interprété, et non pas interprété ligne à ligne, tout du moins dans un premier temps.
Ainsi, nous passerons par une phase de pré-compilation afin d'analyser le code source stibbons. Nous aurons donc un schéma du type~:
code source $\rightarrow$ tokenizer $\rightarrow$ analyse et interprétation

Nous définirons dans un premier temps un langage élémentaire, contenant les affectations ainsi que les instructions de bases pour les tortues (forward, turn-right, etc.) et augmenterons la complexité du langage dans le temps.
Il faudra ainsi définir très rapidement une grammaire, l'idéal étant pour la fin de la semaine. Cette tache est assigné à Florian et Clément.

Pendant ce temps-là, Adrien et Julia sont chargés de définir le modèle. On utilisera pour cela le langage UML, ainsi que le logiciel Umbrello qui permet la génération de code.

Nous avons également abordé la question de la gestion de la bibliographie. Nous utiliserons bibtex dans le rapport pour celle-ci.

\section{16 mars 2015}

Bilan du premier sprint, par rapport à ce qui était prévu, nous notons les différences suivantes :
\begin{itemize}
\item le chargement et l'importation n'est pas complet, on doit passer par le terminal~;
\item l'interprétation d'un agent se fait, mais limité.
\end{itemize}
Pour ce qui est de l'estimation des temps des taches, nous étions globalement au-dessus.
Nous avons ajouter quatres nouvelles tâches au backlog~:
\begin{itemize}
\item ajout de fonctions~;
\item ajout des variables~;
\item ajout des boucles~;
\item ajout des conditionelles.
\end{itemize}
Pour la version 0.2 nous avons choisis les tâches suivantes~:
\begin{itemize}
\item l'utilisateur utilise une variable~;
\item l'utilisateur utilise une fonction~;
\item les tortues fonctionnent en parallèle~;
\item ajout de l'utilisation de breed.
\end{itemize}
Nous avons définit les test pour ces tâches.\\
Nous avons également éffectué les estimations de la durée de chaque tâche en écrivant chacun le temps minimum et le temps maximum estimé, puis nous avons fait la moyenne de ces temps, en discutant des raisons des résultats.
Nous estimons donc le deuxième sprint à deux semaines, à raison de cinq aprés-midi de cinq heures par personne (environ 96 heures au total).\\


\section{17 mars 2015}

Nous avons effectué une réunion avec notre encadrant suite à la fin de notre premier sprint. Certain points on était soulevé tels que la communication inter-agents et le moyen de l'implémenté mais cela concerne la version 0.3, nous devons donc commencé à y réfléchir sans le réaliser.
\\
Le multi-agents de la version 0.2 sera réalisé à l'aide de threads, avec un thread pour chaque tortue.
\\
Le diagramme UML a également été ré-étudié, nous avons donc remarqué certaines différences avec l'implémentation réalisé au cours de la version 0.1, notamment concernant les classes~:
\begin{itemize}
\item Value~;
\item Type~;
\item Colored~;
\item Boolean~;
\item Color~;
\item Les classes liées à l'interpréteur (Tree,~Interpreter).
\end{itemize}


Certaines conception ont été revu~:
\begin{itemize}
\item La classe Shape devrai être relié à une Breed, pas à une Tortue : la Breed définira le comportement d'une tortue~;
\item La classe Breed : renommage en TurtleClass~;
\item Les agrégations et compositions entre les classes sont à rectifier.
\end{itemize}


Pour terminer, deux pistes concernant la gestion de l'arbre syntaxique ont été évoqué~:
\begin{itemize}
\item Réaliser un arbre de dérivation avec chaque frère droit qui est un instruction~;
\item Un arbre abstrait en UML.
\end{itemize}

\section{7 avril 2015}
Nous avons fini le second sprint. Nous avons surtout ajouté des choses dans le modèle et dans l'interpréteur :
il s'agissait de la prise en charge du multi-threading et des variables.\\
Coté modèle, on a ajouté les classes Function, Breed, adapté la classe Turtle, ajouté des mutexs dans chaque classe qui le demandait.
Nous avons également rajouté un systéme de "parenté", qui permet à une tortue de connaître son parent et d'avoir accès à ses propriétés.\\
Coté interprète, il a fallu mettre en place le mettre en place les threads, un à chaque "new agent", et les fonctions.
Lors de cette dernière réunion, nous avons discuté de comment afficher les résultats de simulation de notre programme : diagramme, variables tracées avec un journal... Il faudra choisir une simulation de NetLogo, la tester et comparer les résultats des deux applications.\\
Pour le moment, nous devons redémarrer le programme pour lancer un nouveau fichier. Il faudra utiliser yywrap, qui chaîne les fichiers.\\
Pour la taille du monde, nous pensons à faire une taille par défaut si l'utilisateur n'en choisit pas, et lui donné la possibilité d'en choisir une en début de fichier grâce à une syntaxe différente du stibbons.


\bibliography{doc/report/biblio}
\bibliographystyle{apalike}

\end{document}
