\section{Logo}
\label{Logo}

Le Logo est un langage de programmation apparu dans les années 60 dont l'objectif était alors de permettre à des personnes possédant peu de connaissances en informatique et en programmation (des enfants par exemple) de découvrir ce domaine de manière ludique et interactive. Le langage permettait de contrôler une tortue (un robot) capable d'avancer, de tourner, et d'abaisser un crayon ou un feutre pour dessiner sur une feuille placée au sol. Les instructions entrées permettaient ainsi de tracer des formes, permettant une représentation très visuelle du code (la tortue physique est remplacée dans les implémentations modernes par une tortue virtuelle).

Le langage Logo en lui-même est un dérivé du Lisp (il est d'ailleurs parfois nommé «~Lisp sans parenthèses~») et possède deux types de données~: les mots (chaînes de caractères) et les listes.

Du fait du public visé, les instructions de base (par exemple \verb|forward|, \verb|left|, \verb|pendown|, etc.) et les structures du type procédures, boucles ou conditionnelles sont écrites de façon à être clairement explicites (cf. \ref{logo-proc}, \ref{logo-rpt} et \ref{logo-condi}).
Cependant, comme expliqué dans l'article \cite{Logo}, des études sur Logo ont montré que les enfants, hormis quelques exceptions, n'arrivent pas à créer un programme entier et codent «~ligne par ligne~» ce qui les empêche de créer un modèle complexe et de cerner l'ensemble de la syntaxe de Logo.

\begin{lstlisting}[language=Logo,label=logo-proc,caption=Procédure en Logo]
to <nom de la procédure> :<paramètre>
  <instructions>
  output <valeur à retourner>
end
\end{lstlisting}

\begin{lstlisting}[language=Logo,label=logo-rpt,caption=Boucle en Logo]
repeat <nb fois> [liste d'instructions]
\end{lstlisting}

\begin{lstlisting}[language=Logo,label=logo-condi,caption=Conditionnelles en Logo]
if <test> [liste d'instructions si vrai]
ifelse <test> [liste d'instructions si vrai] [liste d'instructions si faux]
\end{lstlisting}

Les instructions amènent la tortue à se déplacer suivant une certaine distance et un certain angle. On l'oriente ainsi suivant des coordonnées polaires.
