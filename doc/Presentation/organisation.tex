\begin{frame}
\frametitle{Organisation}
Méthode agile : Scrum
\begin{itemize}
\item Temps divisé en sprints
\item Backlog
\end{itemize}
\end{frame}


\begin{frame}
\frametitle{Backlog}
{\tiny
\begin{longtable}[c]{|c|p{1cm}|c|p{1.7cm}|*{4}{c|}}
\hline
\bf id & \bf Scénario utilisateur & \bf Priorité & \bf Tests & \bf Estimation & \bf Sprint & \bf Statut & \bf Temps réel \\
\hline
\endfirsthead
\hline
\bf id & \bf Scénario utilisateur & \bf Priorité & \bf Tests & \bf Estimation & \bf Sprint & \bf Statut & \bf Temps réel \\
\hline
\endhead
1 & L'utilisateur écrit du code dans un éditeur & 200 & L'utilisateur écrit du code dans un éditeur intégré & & & & \\
\hline
2 & L'utilisateur importe du code dans le logiciel & 1100 & Importer du code Stibbons depuis un fichier, vérifier que le code obtenu est bien identique à celui du fichier. & 4h & 1 & Fini & 1h \\
\hline
3 & L'utilisateur visualise les rapports d'erreurs du code & 400 & Exécuter pd 50 et constater une erreur. & 8h & 4 & Fini & 8h \\
\hline
4 & L'utilisateur visualise l'évolution du modèle & 1400 & Vérifier que l'interprétation d'instructions données fait bien évoluer comme prévu la tortue dans son environnement. & 24h & 1 & Fini & 70h \\
\hline
\end{longtable}}
\end{frame}

\begin{frame}
Outils~:
\begin{itemize}
\item CppUnit
\includegraphics[scale=0.16]{doc/Presentation/image/cppUnit.pdf}
\item Json Spirit
\item Git
\includegraphics[scale=0.08]{doc/Presentation/image/git.pdf}
\item GitLab
\includegraphics[scale=0.08]{doc/Presentation/image/gitlab.pdf}
\end{itemize}
\end{frame}
