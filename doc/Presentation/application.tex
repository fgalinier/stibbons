\subsection{Application graphique}

\begin{frame}
\frametitle{Application graphique v0.1}
\begin{center}
\includegraphics[scale=0.16]{doc/Presentation/screenshot/stibbons-0-1-2.png}
\end{center}

\begin{itemize}
	\item Une tortue par défaut
	\item Tortue = triangle noir creux
	\item La tortue peut tracer des lignes
	\item Origine dans le coin supérieur gauche
	\item Impossible d'ouvrir plusieurs programmes
\end{itemize}
\end{frame}

\begin{frame}
\frametitle{Application graphique v0.2}
\begin{center}
\includegraphics[scale=0.16]{doc/Presentation/screenshot/stibbons-0-2-2.png}
\end{center}

\begin{itemize}
	\item Possibilité de créer d'autres tortues
	\item Origine centrée
\end{itemize}
\end{frame}

\begin{frame}
\frametitle{Application graphique v0.3}
\begin{center}
\includegraphics[scale=0.16]{doc/Presentation/screenshot/stibbons-0-3-2.png}
~~~~~~~~
\includegraphics[scale=0.16]{doc/Presentation/screenshot/stibbons-0-3-3.png}
\end{center}

\begin{itemize}
	\item Monde borné, centré
	\item Tortues colorées, dessinées pleines
	\item Lignes colorées
	\item Zones visibles et colorées
	\item Performances de dessin améliorées
\end{itemize}
\end{frame}

\begin{frame}
\frametitle{Application graphique v0.4}
\begin{center}
\includegraphics[scale=0.16]{doc/Presentation/screenshot/stibbons-0-4-2.png}
\end{center}

\begin{itemize}
	\item Plus de tortue par défaut
	\item Ouverture de plusieurs fichiers
	\item Démarrage, pause et redémarrage
	\item Gestion de la vitesse
	\item Export du modèle
	\item Bords rebouclants en option
\end{itemize}
\end{frame}

\begin{frame}
\frametitle{Application graphique v1.0}
\begin{center}
\includegraphics[scale=0.16]{doc/Presentation/screenshot/stibbons-0-5-2.png}
~~~~~~~~
\includegraphics[scale=0.16]{doc/Presentation/screenshot/stibbons-0-5-3.png}
\end{center}

\begin{itemize}
	\item Editeur de programmes Stibbons
	\item Ajout de raccourcis clavier
	\item Bords rebondissants en option
\end{itemize}
\end{frame}

\begin{frame}
\frametitle{Application graphique v1.1}
\begin{center}
\includegraphics[scale=0.16]{doc/Presentation/screenshot/stibbons-0-5-2.png}
~~~~~~~~
\includegraphics[scale=0.16]{doc/Presentation/screenshot/stibbons-0-5-3.png}
\end{center}

\begin{itemize}
	\item Correction d'une grave fuite mémoire
	\item Correction d'erreurs sémantiques
	\item Correction de la coloration syntaxique
\end{itemize}
\end{frame}

\subsection{Application CLI}

\begin{frame}[fragile]
\frametitle{Application CLI}
\begin{center}
\includegraphics[scale=0.24]{doc/Presentation/screenshot/stibbons-cli.png}
\end{center}

\verb|stibbons-cli [options] fichier|

Options~:
\begin{description}
	\item[\texttt{-{}-}export s] Exporte le modèle toutes les \emph{s} secondes
	\item[\texttt{-{}-}prefix p] Préfixe les fichiers exportés avec \emph{p}
	\item[\texttt{-{}-}png] Génère une image PNG pour chaque export
	\item[\texttt{-{}-}no-json] N'exporte pas le modèle dans un fichier JSON
\end{description}

Arguments~:
\begin{description}
	\item[fichier] Le fichier de programme Stibbons à exécuter
\end{description}

\end{frame}

